\newpage
\begin{center}
  \textbf{\large Введение}
\end{center}
\refstepcounter{chapter}
\addcontentsline{toc}{chapter}{Введение}


Данная работа посвящена анализу нуклеотидных последовательностей генов семейства \textit{Nxf} у представителей различных филогенетических групп.
Она является прямым продолжением бакалаврской работы, выполненной на кафедре генетики и биотехнологии СПбГУ в лаборатории генетики животных.

Ген \textit{Nxf1} является нашим объектом интереса в связи с тем, что возможно образование транскрипта с сохраненным интроном в нем, что само по себе является нечастым явлением.
Более того, в данном так называемом "кассетном интроне" присутствует стоп-кодон, и несмотря на это, показано, что данный транскрипт необходим для правильного формирования нервной ткани у \textit{Drosophila melanogaster}.

В предыдущей работе были проанализированы нуклеотидные последовательности данного гена у видов из группы Arthropoda, почти половина из которых являлись представителями семейства Drosophilidae.
Сейчас мы хотим сосредоточить наше внимание на группу Ctenophora, в связи с тем, что их систематическое положение на данный момент под вопросом.
Мы предполагаем, что использование гена \textit{Nxf1} в качестве маркерного поможет разрешить данный вопрос.

Следовательно, актуальность работы заключается в потенциальном разрешении гипотезы о дважды независимом возникновении нервной системы у гребневиков, а также разрешении их систематического положения с использованием Nxf1 в качестве гена-маркера.
Кроме того, подробное изучение нуклеотидной последовательности гена и "кассетного" интрона в частности, позволит проследить его эволюцию в лучшем случае, начиная от прокариот.

\textbf{Целью} данной работы является изучение структуры гена Nxf1 у представителей разных филогенетических групп для выявления закономерностей эволюции генов семейства Nxf.

В рамках цели сформулированы следующие \textbf{задачи}:

\begin{enumerate}
    \item Разработка пайплайна для автоматизации обработки и анализа данных
    \begin{itemize}
        \item использование нуклеотидных последовательностей представителей семейства Drosophilidae с целью разработки метода автоматизированного анализа результатов BLAST
    \end{itemize}
    \item Поиск
    \begin{itemize}
        \item "консервативной кассеты" в нуклеотидной последовательности гена \textit{Nxf1} у представителей Opisthokonta в геномных базах данных
        \item полной нуклеотидной последовательности гена \textit{Nxf1} у представителей Ctenophora
        \item эволюционного предшественника гена \textit{Nxf1} у простейших
        \item РНК-связывающих участков у прокариот
    \end{itemize}
    \item Сравнение найденных нуклеотидных последовательностей с
    \begin{itemize}
        \item полной нуклеотидной последовательностью гена \textit{Nxf1}
        \item "консервативной кассетой" в составе гена \textit{Nxf1}
        \item "кассетным" интроном в составе "консервативной кассеты" гена Nxf1 между представителями разных филогенетических групп
    \end{itemize}
\end{enumerate}

На данный момент ведется активная работа над разработкой пайплайна для обработки результатов BLAST.
