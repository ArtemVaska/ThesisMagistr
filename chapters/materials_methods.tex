\newpage
\begin{center}
  \textbf{\large Материалы и методы}
\end{center}
\refstepcounter{chapter}
\addcontentsline{toc}{chapter}{Материалы и методы}


В качестве \textbf{материалов} выступают нуклеотидные последовательности представителей разных филогенетических групп, взятые из открытых и закрытых баз данных.

Для выполнения вышестоящих задач планируется использование следующих \textbf{методов}:

\begin{enumerate}
    \item Поиск ортологов внутри таксономических групп с помощью BLAST или Diamond
    \item Анализ и обработка полученных результатов путем выполнения скриптов, написанных на языке программирования Python, с использованием различных библиотек (например, Biopython)
    \item Использование командной строки Linux и различных программ обработки и визуализации данных
    \begin{itemize}
        \item для построения пайплайнов – SnakeMake
        \item для выравнивания последовательностей на локальном компьютере – BWA (Burrows-Wheeler Alignment)
        \item для визуализации полученных выравниваний – IGV (Integrative Genomics Viewer)
    \end{itemize}
    \item Множественное выравнивание нуклеотидных последовательностей с помощью алгоритма MUSCLE в программах UNIPRO UGENE и MEGA-X
\end{enumerate}

Пайплайн для обработки результатов BLAST включает следующие этапы:

\begin{enumerate}
    \item Получить и прочесть результаты BLAST
    \item Посчитать Query Coverage (QC) в каждой найденной последовательности
    \item Отфильтровать по заданному порогу для QC TODO
    \item Сгруппировать оставшиеся с помощью алгоритма кластеризации TODO
    \item Загрузить найденные последовательности в соответствующие папки
    \item Провести множественные выравнивания TODO
    \item Проанализировать итоговые результаты TODO
\end{enumerate}

Пункты, помеченные "TODO" находятся на этапе разработки. С реализацией остальных стадий анализа можно ознакомиться в GitHub репозитории:

https://github.com/ArtemVaska/Diploma/tree/bachelor\_repeat

в ноутбуке bachelor\_repeat.ipynb в папке Scripts.
