\documentclass[
candidate, % тип документа
subf, % подключить и настроить пакет subfig для вложенной нумерации рисунков
times % шрифт Times как основной
]{disser}

\renewcommand{\rmdefault}{ftm} % Основная строчка, которая позволяет получить шрифт Times New Roman

\usepackage[T2A]{fontenc}
\usepackage[left=3cm, right=1cm, top=2cm, bottom=2cm]{geometry}
\usepackage{graphicx}
\usepackage[utf8]{inputenc}
\usepackage[english,russian]{babel}
\usepackage{pdfpages}
\ifpdf\usepackage{epstopdf}\fi
\usepackage{array}
\newcommand{\PreserveBackslash}[1]{\let\temp=\\#1\let\\=\temp}
\newcolumntype{C}[1]{>{\PreserveBackslash\centering}p{#1}}
\usepackage{tabularx}
\usepackage{dcolumn}
\usepackage{bm}
\usepackage{hyperref}
\usepackage{color}
\usepackage{epstopdf}
\usepackage{amsmath}
\usepackage{amssymb}
\usepackage{cite}
\usepackage{multirow}
\usepackage{afterpage}
\usepackage[font={normal}]{caption}
\usepackage{amsmath} % align
\usepackage[onehalfspacing]{setspace}% 1,5 интервал
\usepackage{fancyhdr} % пакет для установки колонтитулов
\usepackage{listings}

\pagestyle{fancy} % смена стиля оформления страниц
\fancyhf{} % очистка текущих значений
\fancyfoot[C]{\thepage} % установка верхнего колонтитула
\renewcommand{\headrulewidth}{0pt} % убрать разделительную линию
\captionsetup{format=hang,labelsep=period}
% Использовать полужирное начертание для векторов
\let\vec=\mathbf

\setcounter{tocdepth}{2} % Номер страницы, с которой начинается нумерация
\graphicspath{{images/}} % Папки, в которых ищутся картинки

\pagestyle{footcenter}
\chapterpagestyle{footcenter}
